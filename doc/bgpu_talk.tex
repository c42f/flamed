\documentclass{beamer}

%\usetheme{Boadilla}

\usepackage{graphics}
\usepackage{amsmath}

\usefonttheme{structurebold}
\useinnertheme{default}
\useoutertheme{smoothbars}

%--------------------------------------------------
% hacked albatross color theme
\setbeamercolor*{normal text}{fg=yellow!20!white,bg=blue!0!black}

\setbeamercolor*{example text}{fg=green!65!black}

\setbeamercolor*{structure}{fg=blue!25!white}

\setbeamercolor{palette primary}{use={structure,normal text},fg=structure.fg,bg=normal text.bg!75!black}
\setbeamercolor{palette secondary}{use={structure,normal text},fg=structure.fg,bg=normal text.bg!60!black}
\setbeamercolor{palette tertiary}{use={structure,normal text},fg=structure.fg,bg=normal text.bg!45!black}
\setbeamercolor{palette quaternary}{use={structure,normal text},fg=structure.fg,bg=normal text.bg!30!black}

\setbeamercolor*{block body}{bg=normal text.bg!90!black}
\setbeamercolor*{block body alerted}{bg=normal text.bg!90!black}
\setbeamercolor*{block body example}{bg=normal text.bg!90!black}
\setbeamercolor*{block title}{parent=structure,bg=normal text.bg!75!black}
\setbeamercolor*{block title alerted}{use={normal text,alerted text},fg=alerted text.fg!75!normal text.fg,bg=normal text.bg!75!black}
\setbeamercolor*{block title example}{use={normal text,example text},fg=example text.fg!75!normal text.fg,bg=normal text.bg!75!black}

\setbeamercolor{item projected}{fg=black}

\setbeamercolor*{sidebar}{parent=palette primary}

\setbeamercolor{palette sidebar primary}{use=normal text,fg=normal text.fg}
\setbeamercolor{palette sidebar secondary}{use=structure,fg=structure.fg}
\setbeamercolor{palette sidebar tertiary}{use=normal text,fg=normal text.fg}
\setbeamercolor{palette sidebar quaternary}{use=structure,fg=structure.fg}

\setbeamercolor*{separation line}{}
\setbeamercolor*{fine separation line}{}



%\usecolortheme{orchid} % inner color theme
%\usecolortheme{seahorse} % outer color theme
%\usecolortheme{albatross}
%\usecolortheme[rgb={0.7,0.4,0.1}]{structure}
%\usecolortheme[hsb={0.6,0.4,0.55}]{structure}
%\usecolortheme[hsb={0.6,0.4,0.45}]{structure}
%\setbeamercolor{background canvas}{bg=}

\usefonttheme[onlymath]{serif}
%\usefonttheme[stillsansseriftext,stillsansserifsmall]{serif}

\setbeamercovered{transparent}

\setbeamertemplate{navigation symbols}{}

\title[Interactive fractal flames]{Interactive fractal flames with CUDA and OpenGL}
%\titlegraphic{\includegraphics[viewport=8 653 558 831,width=6cm]{figures/logos.pdf}}
%\subtitle{}


\author{Chris Foster}
\institute{ROAMES}
\date{March 22, 2011}


%===============================================================================
%===============================================================================
\begin{document}

\begin{frame}[plain]
  \titlepage
\end{frame}

%\begin{frame}
%  \frametitle{Overview}
%  \tableofcontents
%\end{frame}


%===============================================================================
\section{Introduction}

\begin{frame}
  \frametitle{What is a fractal?}
  \begin{itemize}
    \item<1>
      Fractional (Hausdorff) dimension $>$ topological dimension
    \item<2>
      Self similar
    \item<3>
      Power law scaling behaviour
  \end{itemize}
  \vspace{4cm}
\end{frame}



%===============================================================================
\section{Mathematics}

\begin{frame}
  \frametitle{Iterated function systems}
  \begin{itemize}
    \item
      Set of contractive functions $F_i \colon \mathbb{R}^2\to\mathbb{R}^2$
    \item
      Attractor obeys recursive set equation
      \[
      A = \bigcup_{i=1}^N F_i(A)
      \]
    \item
      $F_i$ traditionally \emph{affine} (rotation/translation/scaling)
  \end{itemize}
\end{frame}


\begin{frame}
  \frametitle{Flame Fractals}
  \begin{itemize}
    \item
      Non-affine $F_i$: more artistic flexibility
      \[
      F_i = P_i\bigg(\sum_m v_{im} V_m\big(Q_i(x,y)\big)\bigg)
      \]
    \item
      Maps $P_i$ and $Q_i$ are affine, eg:
      \[
        Q_i(x,y) = \big(a_i x + b_i y + c_i,\; d_i x + e_i y + f_i \big)
      \]
    \item
      $V_m$ are nonlinear ``variations''
  \end{itemize}
\end{frame}


\begin{frame}
  \frametitle{Algorithm}
  \begin{itemize}
    \item
      Monte Carlo sampling
    \item
      Maps choosen at random
    \item
      Color mixed in
  \end{itemize}
\end{frame}


%===============================================================================
\section{Implementation}

\begin{frame}
  \frametitle{Implementation overview}
  \begin{itemize}
    \item
      Generate point list using CUDA
    \item
      Render with additive OpenGL blending
    \item
      HDR tone map and gamma correction
  \end{itemize}
\end{frame}


\begin{frame}
  \frametitle{CUDA}
  \begin{itemize}
    \item
      Warp divergence
    \item
      curand
  \end{itemize}
\end{frame}


\begin{frame}
  \frametitle{OpenGL HDRI}
  \begin{itemize}
    \item
      Tone mapping
    \item
      $RGB_{lin}\to xyY \to xyY' \to RGB_{lin} \to RGB_{gam}$
    \item
      Gamma correction
  \end{itemize}
\end{frame}



%===============================================================================
\section*{End}

\begin{frame}
  \frametitle{Thanks (collage theorem)}
\end{frame}

\begin{frame}
  \frametitle{References}
  \begin{itemize}
    \item
      S. Draves and E. Reckase, \textit{The Fractal Flame Algorithm}, 2003.
    \item
      C. Schied \textit{et al.}, \textit{High-Performance Iterated Function 
      Systems}, in GPU computing gems, Emerald Edition, 2011.
    \item
      K. J. Falconer, \textit{Fractal Geometry: Mathematical Foundations and 
      Applications}, Wiley \& Sons, Chichester, 2003.
%    \item
%      Barnsley, \textit{Fractals Everywhere}
  \end{itemize}
%  \nocite{...}
%  \bibliographystyle{unsrt}
%  %\twocolumn
%  \fontsize{5pt}{6pt}
%  \bibliography{../bibfiles/general_cjf}
%  \normalsize
\end{frame}



%===============================================================================
%===============================================================================
\part{Extras}

\begin{frame}
  \frametitle{Collage Theorem}
\end{frame}


\end{document}
